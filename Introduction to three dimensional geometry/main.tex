\documentclass[12pt]{article}
\usepackage{graphicx} % Required for inserting images
\usepackage[margin=2cm]{geometry}
\usepackage{multicol,amsmath, amssymb}
\usepackage{xcolor}
\usepackage{titlesec}
\usepackage{pgfplots}

\titleformat{\subsection}
{\color{red}\normalfont\large\bfseries}
{\thesubsection}{1em}{}

\titleformat{\subsubsection}
{\color{blue}\normalfont\large\bfseries}
{\thesubsubsection}{1em}{}

\titlespacing{\subsection}{0pt}{0pt}{0pt} % Adjust the spacing here
\titlespacing{\subsubsection}{0pt}{\baselineskip}{0pt} % Adjust the spacing here

\begin{document}
\begin{center}
    {\LARGE \textbf{Introduction to Three Dimensional Geometry} }
\end{center}

    To refer to a point in space we require a third axis (say z-axis) which leads to the concept of three-dimensional geometry. In this chapter, we study the basic concept of geometry in three-dimensional space.

    \subsection*{Octant}

Consider three mutually perpendicular planes meet at a point O. 
\begin{center}
    \includegraphics*[scale=0.7]{1.png}
\end{center}


Let these three planes intercept along three lines XOX’, YOY’ and ZOZ’ called the x-axis, y-axis, and z-axis respectively.The three coordinate planes divide the space into eight parts known
as octants. These octants could be named as XOYZ, X'OYZ, X'OY'Z, XOY'Z, XOYZ',
X'OYZ', X'OY'Z' and XOY'Z'. and denoted by I, II, III, ..., VIII , respectively.

\begin{itemize}
    \item coordinate of the poin in x axis is of the form $(x,0,0)$
    \item coordinate of the poin in y axis is of the form $(0,y,0)$
    \item coordinate of the poin in z axis is of the form $(0,0,z)$
    \item coordinate of the poin in xy plane is of the form $(x,y,0)$
    \item coordinate of the poin in yz plane is of the form $(0,y,z)$
    \item coordinate of the poin in xz plane is of the form $(x,0,z)$
\end{itemize}
The sign of the coordinates of a point determine the octant in which the
point lies. The following table shows the signs of the coordinates in eight octants.

\begin{center}
    \includegraphics*[scale=0.7]{2.png}
\end{center}
\subsubsection*{Distance between Two Points}
Distance between two points $P(x_1,y_1)$ and $Q(x_2,y_2)$ in 3D plane is  
$$PQ=\sqrt{(x_1-x_2)^2+(y_1-y_2)^2+(z_1-z_2)^2}$$



\end{document}