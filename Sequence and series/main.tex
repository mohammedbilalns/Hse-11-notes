
\documentclass[12pt]{article}
\usepackage{graphicx} % Required for inserting images
\usepackage[margin=2cm]{geometry}
\usepackage{multicol,amsmath, amssymb}
\usepackage{xcolor}
\usepackage{titlesec}
\usepackage{pgfplots}

\titleformat{\subsection}
{\color{red}\normalfont\large\bfseries}
{\thesubsection}{1em}{}

\titleformat{\subsubsection}
{\color{blue}\normalfont\large\bfseries}
{\thesubsubsection}{1em}{}

\titlespacing{\subsection}{0pt}{0pt}{0pt} % Adjust the spacing here
\titlespacing{\subsubsection}{0pt}{\baselineskip}{0pt} % Adjust the spacing here

\begin{document}
\begin{center}
    {\LARGE \textbf{SEQUENCES AND SERIES} }
\end{center}

\begin{multicols}{2}

    A \textbf{sequence} can be regarded as a function whose domain is the set of natural numbers or some subset of it of the type $\{1, 2, 3, ….., k\}$.
    Generally denoted by $a_1, a_2, \dots, a_n,\dots$

    Let $a_1, a_2, \dots, a_n,\dots$ be a sequence. Then the expression $a_1 + a_2 + \dots+ a_n + \dots$ is called the \textbf{series} associated with the given sequence.

    \subsection*{Arithmetic Progression (A.P.)}
    A sequence $a_1, a_2, \dots, a_n,\dots$ is called an \textbf{arithmetic sequence} or arithmetic progression if $a_{n+1} = a_2 + d, n \in N$, where $a_1$ is called the first term and the constant term $d$ is called the common difference of the AP.

    \begin{itemize}
        \item $a, a + d, a + 2d, \dots$ where $a$ is the first term and $d$ is a common difference is the \textbf{standard form} of AP
        \item If a constant is added to each term of an AP, the resulting sequence is also an AP.

        \item If a constant is subtracted to each term of an AP, the resulting sequence is also an AP.
        
        \item If each term of an AP is multiplied by a constant $k$, the resulting sequence is also an AP. But the resulting AP will have a common difference $kd$.
        
        \item If each term of an AP is divided by a constant $k$, the resulting sequence is also an AP. But the resulting AP will have a common difference $\frac{d}{k}$.
    
    
   \item  \textbf{$n^{th}$ term (general term)} of the A.P. is 
    $$a_n = a +(n-1) d$$
    \item Sum of first n terms of the AP is 
    $$S_n = \frac{n}{2}[2a+(n-1)d]$$
    $$S_n=\frac{n}{2}[a_1+a_n]$$
    \item Arithmetic mean between a and b is $\frac{a+b}{2}$
\end{itemize}

    \subsection*{Geometric Progression(G.P)}
    A sequence $a_1, a_2, \dots, a_n,\dots$ is called \textbf{Geometric sequence} or Geometric progression if $\frac{a_{k+1}}{a_k}=r , k \geq 1$, where $a_1$ is called the first term and the constant term $r$ is called the common ratio of the GP
    \begin{itemize}
        \item $a,ar,ar^2,ar^3,\dots$ where a is the first term and r is the common ratio is the \textbf{standard form} of GP
        \item $n^{th}$ term of the GP is $t_n=ar^{n-1}$
        \item Sum of n terms 
         $$S_n=\frac{a(1-r^n)}{1-r}, r<1$$
         $$S_n = \frac{a(r^n -1)}{r-1}, r >1$$
    \item Geometric mean between $a$ and $b$ is $\sqrt{ab}$
    \item Arithmetic mean $\geq$ Geometric mean
    \item Sum of an infinite GP $a_1, a_2, \dots, a_n,\dots$ is 
    $$S_{\infty} =\frac{a}{1-r}$$
    \end{itemize}

    \subsubsection*{Sum to n terms of Special Series}

    \begin{itemize}
        \item $1+2+3+4+\dots = \frac{n(n+1)}{2}$
        \item $1^2+2^2+3^2+4^2+\dots = \frac{n(n+1)(2n+1)}{6}$
        \item $1^3+2^3+3^3+4^3+\dots = [\frac{n (n+1)}{2}]^2$
    \end{itemize}



    
\end{multicols}
\end{document}