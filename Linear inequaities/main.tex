
\documentclass[12pt]{article}
\usepackage{graphicx} % Required for inserting images
\usepackage[margin=2cm]{geometry}
\usepackage{multicol,amsmath, amssymb}
\usepackage{xcolor}
\usepackage{titlesec}
\usepackage{pgfplots}

\titleformat{\subsection}
{\color{red}\normalfont\large\bfseries}
{\thesubsection}{1em}{}

\titleformat{\subsubsection}
{\color{blue}\normalfont\large\bfseries}
{\thesubsubsection}{1em}{}

\titlespacing{\subsection}{0pt}{0pt}{0pt} % Adjust the spacing here
\titlespacing{\subsubsection}{0pt}{\baselineskip}{0pt} % Adjust the spacing here

\begin{document}
\begin{center}
    {\LARGE \textbf{LINEAR INEQUALITIES} }
\end{center}

\begin{multicols}{2}

    Two real numbers or two algebraic expressions related by the symbols $<, >, \leq$ or $\geq$ form an inequality. In this unit we study linear inequalities in one and two variables, their formation and solution graphically.

\subsection*{ Linear Inequalities in One Variable}

\subsubsection*{Examples}
\begin{itemize}
    \item $30x<200$
    \item $5x - 3 < 3x +1$
    \item $\frac{5-2x}{3}<\frac{x}{6}-5$
\end{itemize}

The \textbf{solution} of an inequality in one variable is a value of the variable $x$ which makes it a true statement.

\subsubsection*{Rules}
\begin{enumerate}
    \item Equal numbers may be added to (or subtracted from) both sides of an equation.
    \item Both sides of an equation may be multiplied (or divided) by the same non-zero
    number.
    
    \item If we multiply or divide both sides of an inequation by a negative number, the inequality sign will be reversed.
    \item To represent $x < a$ (or $x > a$) on a number line, put a circle on the number $'a'$ and a dark line to the left (or right) of the number $'a'$.
    \item To represent $x \leq a$ (or $x \geq a$) on a number line, put a dark circle on the number $'a'$ and a dark line to the left (or right) of the number $'a'$. 
\end{enumerate}
\includegraphics*[scale=0.7]{1.png}
\subsection*{Graphical Solution of Linear Inequalities in Two Variables}

The region containing all the solutions of an inequality is called the \textbf{solution region}.

\begin{itemize}
    \item To find the solution of inequalities ,First we find the line $ax+by=c$ 
    \item In order to identify the half-plane represented by inequality, it is just sufficient to take any point $(a, b)$ [say point $(0, 0)$] not on the line and check whether it satisfies the inequality or not. If it satisfies, then the inequality represents the half-plane and shade the region which contains the point, otherwise, the inequality represents that half-plane which does not contain the point within it.
    \item If the inequality is of the type $ax + by \geq c$ or $ax + by \leq c$, then the point on the line $ax + by = c$ is also included in the solution. So draw a dark line in the solution region.
    \item If the inequality is of the type $ax + by > c$ or $ax + by < c$, then the point on the line $ax + by = c$ are not to be included in the solution. So draw a broken or dotted line in the solution region.
\end{itemize}

\end{multicols}
\end{document}